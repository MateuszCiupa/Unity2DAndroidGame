\documentclass[12pt, a4paper]{article}
\usepackage{polski}
\usepackage[utf8]{inputenc}
\usepackage{fancyhdr}
\usepackage{lastpage}
\usepackage[none]{hyphenat}

\pagestyle{fancy}
\fancyhf{}

\rhead{\textit{All About Survival}}
\lhead{Specyfikacja funkcjonalna}
\cfoot{\thepage/\pageref{LastPage}}
\setlength{\headheight}{15pt}

\author{Mateusz Ciupa 291062}
\title{Specyfikacja implementacyjna gry mobilnej \\\textit{All About Survival}} 

\begin{document}

\maketitle
\tableofcontents

\section{Założenia projektowe}
Gra \textit{All About Survival} jest grą mobilną na urządzenia z systemem
Android, która powstanie przy użyciu silnika \textit{Unity}, a postacie wraz 
z otoczeniem oraz animacjami przy użyciu narzędzia \textit{Aseprite}. Logika 
gry będzie się opierała na mniej lub bardziej złożonych skryptach napisanych 
w języku \textit{C\#}. Elementy związane z dźwiękiem będą produkowane przy
użyciu programu \textit{Audacity}.

\section{Architektura}
Logika gry będzie się opierała o poniższy zestaw klas$\colon$

\begin{enumerate}

\item \texttt{public class Point}

Klasa odwzorowująca punkt na osi poziomej.

Zestaw pól$\colon$

\begin{itemize}
\item \texttt{public float X}
\item \texttt{public float Y}
\end{itemize}

Zestaw metod$\colon$
\begin{itemize}
\item \texttt{public float getRange(Point A)}

Metoda zwracająca odległość pomiędzy dwoma punktami.
\end{itemize}

\item \texttt{public class Vector}

Klasa odwzorowująca wektor na osi poziomej (punkt początkowy oraz 
przemieszczenie).

Zestaw pól$\colon$

\begin{itemize}
\item \texttt{public Point A}
\item \texttt{public float Dislocation}
\end{itemize}

\item \texttt{public class Character}

Klasa odwzorowująca każdą istotę w grze.

Zestaw pól$\colon$

\begin{itemize}
\item \texttt{public string Name}
\item \texttt{public int CurrentHealth}
\item \texttt{public int MaxHealth}
\item \texttt{public float MoveSpeed}
\item \texttt{public float CurrentPosition}
\end{itemize}

Zestaw klas$\colon$

\begin{itemize}
\item \texttt{public void MoveRight()}

Metoda wywołująca animację poruszania w prawo.

\item \texttt{public void MoveLeft()}

Metoda wywołująca animację poruszania w lewo.

\item \texttt{public void IdleRight()}

Metoda wywołująca animację postaci w bezczynności skierowaną w prawo.

\item \texttt{public void IdleLeft()}

Metoda wywołująca animację postaci w bezczynności skierowaną w lewo.

\end{itemize}

\item \texttt{public class MainCharacter : Character}

Klasa dziedzicząca po klasie \texttt{Character} odwzorowująca główną postać z
gry.

Zestaw pól$\colon$

\begin{itemize}
\item \texttt{public int Coins}
\item \texttt{public int Wood}
\item \texttt{public int Weapon}
\end{itemize}

Zestaw metod$\colon$

\begin{itemize}
\item \texttt{public void ShootRight()}

Metoda wywołująca animację strzelania z posiadanej broni w prawo oraz 
wykonująca odpowiednie akcje przy trafieniu w cel.

\item \texttt{public void ShootLeft()}

Metoda wywołująca animację strzelania z posiadanej broni w lewo oraz 
wykonująca odpowiednie akcje przy trafieniu w cel.

\end{itemize}

\item \texttt{public class Drifter : Character}

Klasa dziedzicząca po klasie \texttt{Character} odwzorowująca wałęsającego
się człowieka, łucznika oraz robotnika.

Zestaw pól$\colon$

\begin{itemize}
\item \texttt{public int Role}
\item \texttt{public bool IsAccessible}
\end{itemize}

Zestaw metod$\colon$

\begin{itemize}
\item \texttt{public void ShootRight()}

Metoda wywołująca animację strzelania w prawo oraz wykonująca odpowiednie 
akcje przy trafieniu w cel.

\item \texttt{public void ShootLeft()}

Metoda wywołująca animację strzelania w lewo oraz wykonująca odpowiednie 
akcje przy trafieniu w cel.

\item \texttt{public void MoveToCheckpoint()}

Metoda, która anuluje wszystkie aktualne akcje postaci oraz powoduje 
przemieszczenie postaci do punktu kontrolnego.

\item \texttt{public void Build(Point A)}

Metoda powodująca przemieszczenie postaci do wyznaczonego miejsca oraz 
wywołanie animacji budowania. 

\end{itemize}

\item \texttt{public class Animal : Character}

Klasa dziedzicząca po klasie \texttt{Character} odwzorowująca zwierzę, będące
głównym środkiem pozyskiwania monet.

Zestaw metod$\colon$

\begin{itemize}
\item \texttt{public void RunAway()}

Metoda wywołująca przemieszczenie zwierzęcia z aktualnej pozycji na niewielką 
odległość. Metoda jest wywoływana w przypadku zbliżenia się głównej postaci 
na bliską odległość do zwierzęcia.

\end{itemize}

\item \texttt{public class Creature : Character}

Klasa dziedzicząca po klasie \texttt{Character} odwzorowująca nieprzyjazną
istotę, której głównym zadaniem jest dostanie się do punktu kontrolnego.

Zestaw metod$\colon$

\begin{itemize}
\item \texttt{public void AttackRight()}

Metoda wywołująca animację atakowania w prawo oraz wykonująca odpowiednie 
akcje przy trafieniu w cel.

\item \texttt{public void AttackLeft()}

Metoda wywołująca animację atakowania w lewo oraz wykonująca odpowiednie 
akcje przy trafieniu w cel.

\end{itemize}

\end{enumerate}

\section{Testy jednostkowe}

Jako że językiem, którym się posługuję do pisania skryptów jest \textit{C
\#}, wykorzystam wbudowaną usługę do przeprowadzania testów, którą oferuje 
Microsoft, zwaną Unit Test Project. Testy jednostkowe będą przeprowadzone na 
publicznych metodach każdej klasy.

\end{document}